\section{DISCUSSION AND CONCLUSIONS 讨论和结论}

\begin{paracol}{2}
    
    The standard penetration test is easy and less expensive to conduct mainly because of its standardization, though it has a drawback in rather low accuracy. Investigation into soil deposits by way of the standard penetration test allows a three-dimensional grasp of their structure, if the test is conducted as many times as possible, so that as many test data as possible can be obtained.

    \switchcolumn

    标准渗透测试容易进行且成本较低,这主要是由于其标准化,尽管它具有准确性较低的缺点。 通过标准渗透试验对土壤沉积物进行调查,可以对结构进行三维掌握,如果该试验进行了尽可能多的次数,则可以获得尽可能多的试验数据。  
    
    \switchcolumn*

    With due attention paid to such advantageous feature of the standard penetration test, the authors have finally succeeded in introducing \autoref{equation:9} and \autoref{equation:11} by arranging in a statistical way the data on initial shear modulus $G_0$, shear strength $S_u$ and N-value of the standard penetration test.
    
    \switchcolumn
       
    在充分注意标准渗透试验的这种有利特性的基础上,作者们终于成功地引入了\cnequationref{equation:9}和\cnequationref{equation:11},通过统计方式安排了初始剪切模量$G_0$,剪切强度$S_u$和N值的数据。 标准渗透测试。      
    \switchcolumn*

    On the other hand, the authors have also evaluated the relation between $G_0$ and $S_u$, so that it turns out to be expressed in \autoref{equation:13}. Elimination of N from the \autoref{equation:9} and \autoref{equation:11} allows the relation between $G_0$ and $S_u$ to be expressed in the following equation :

    \switchcolumn
       
    另一方面,作者还评估了$G_0$和$S_u$之间的关系,从而证明它由\cnequationref{equation:13}表示。 从等式\cnequationref{equation:9}和\cnequationref{equation:11}消除N可以使$G_0$和$S_u$之间的关系表示为以下方程式: 

\end{paracol}

\begin{align}
    G_0=487S_u^{0.928}(\rm{kg/m^2})
    \label{equation:15}
\end{align}

\begin{paracol}{2}

    \autoref{equation:15} is represented in \autoref{figure:14}, so that it can be compared with \autoref{equation:13}. A close resemblance existing between the two equations is self-explanatory of mutual relationship among initial shear modulus, shear strength and N-value.

    The investigations covered in this paper are summarized as follows.

    \switchcolumn

    \cnequationref{equation:15}在\cnfigureref{figure:14}中表示,因此可以将其与\cnequationref{equation:13}进行比较。 这两个方程之间存在非常相似的地方,这很容易说明初始剪切模量,剪切强度和N值之间的相互关系。
    本文涵盖的研究概述如下。

    \switchcolumn*

    \begin{enumerate}
        \item Shear strain amplitude generated in a very soft soil deposit by the well-shooting test was in general found less than $10^{-3}\%$. It follows from this that shear modulus obtained by in-situ tests can be regarded as the initial shear modulus.
        \item As a result of investigating the mutual relationship among initial shear modulus, shear strength and N-value, it became clear that the initial shear modulus of cohesive soil is proportional to its shear strength obtained under the $K_0$-condition.
        \item The ratio between the initial shear modulus and the shear strength can be estimated to be about 500 from \autoref{equation:13}and \autoref{figure:15}.
        \item It is simple and convenient if the initial shear modulus is estimated from N-value. However, we should not resort to N-value only, because errors of the standard penetration test become large, especially when N-value is less than 2.
    \end{enumerate}
    
    \switchcolumn

    \begin{enumerate}
        \item 一般情况下,通过射孔试验在非常软的土壤沉积物中产生的剪切应变振幅通常小于$10^{-3}\%$。 由此可知,可以将通过原位测试获得的剪切模量视为初始剪切模量。
        \item 研究了初始剪切模量,剪切强度和N值之间的相互关系,结果发现,黏性土的初始剪切模量与其在$K_0$条件下获得的剪切强度成正比。
        \item 根据\cnequationref{equation:13}和\cnfigureref{figure:15}可以估计出初始剪切模量与剪切强度之间的比率约为500。
        \item 如果根据N值估算初始剪切模量,则既简单又方便。 但是,我们不应仅诉诸N值,因为标准渗透测试的误差会变得很大,尤其是当N值小于2时。
    \end{enumerate}

\end{paracol}