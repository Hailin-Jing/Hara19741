\section{INTRODUCTION 介绍}

\begin{paracol}{2}

    In Japanese cities, for lack of suitable building sites, there has recently been a marked tendency to plan construction of large and tall buildings on soft grounds which were never considered in the past as a building site. Every structure is influenced by the local ground behavior to a considerable extent during earthquakes, especially when built on a soft ground. Since any structure built on such a ground is inevitably subject to considerable displacement, particularly at its foundation, it is desirable that an earthquake-proof design follows a dynamic analysis by means of a simulation model of structure-foundation-ground system. Needless to mention, understanding by experiment of the dynamic properties of soils is required for such a dynamic analysis. These properties are characterized by the following items:
    \begin{enumerate}
        \item Initial shear modulus
        \item Shear strength
        \item Non-linear characteristics
        \item Damping
    \end{enumerate}

    \switchcolumn

    在日本城市中,由于缺乏合适的建筑工地,最近有一种明显的趋势是计划在柔软的地面上建造大型高层建筑,而过去从未将其视为建筑工地。 在地震期间,尤其是在软土地上建造时,每个结构在很大程度上都会受到当地地面行为的影响。 由于在这样的地面上建造的任何建筑物都不可避免地会遭受相当大的位移,特别是在其基础上,因此希望通过结构基础-地面系统的仿真模型对动态设计进行抗震设计。 不用说,进行这种动力学分析需要通过实验来了解土壤的动力学特性。 这些特性的特征如下:
    \begin{enumerate}
        \item 初始剪切模量
        \item 剪切强度
        \item 非线性特性
        \item 阻尼
    \end{enumerate}

    \switchcolumn*

    In order to research the above problems, it is desirable to make the best of the merits of the in-situ test and the laboratory test. By in-situ tests we are able to grasp directly the mechanical properties of soils at natural soil deposit, but the test conditions are not always clear.On the other hand, conditions of the laboratory test are clear, and the laboratory test enables us to obtain the mechanical properties at large deformation, but its weak point is disturbance of soil structure during the sampling and testing processes.
    
    Therefore the in-situ and the laboratory tests have their merits and demerits. In order to take advantage of the merits, the authors took notice of the following items : 

    \switchcolumn

    为了研究上述问题,希望充分利用原位试验和实验室试验的优点。 通过现场试验,我们能够直接掌握天然土壤沉积物的土壤力学性能,但是试验条件并不总是很清楚。另一方面,实验室试验的条件是明确的,并且实验室试验使我们能够获得大变形时的机械性能,但是它的弱点是在采样和试验过程中对土壤结构的干扰。
  
    因此,现场试验和实验室试验各有优缺点。 为了利用这些优点,作者注意到以下几点:

    \switchcolumn*

    \begin{enumerate}
        \item There are many cases where soil stiffness increases in general with strength according to the experience of laboratory or in-situ test.
        \item \citet{Wilson2010419} presented many data on Young's modulus E and undrained shear strength Su by the laboratory test. Furthermore, \citet{DAppolonia19711359} showed a number of data on Young's modulus $E$ obtained from initial settlement curve and undrained shear strength $S_u$. It is recognized that the values of ratio $E/S_u$ calculated from their data are within a certain limited region.
        \item The shear strength of soil can be calculated from the values of effective overburden pressure, angle of internal friction and cohesion, and the angle of internal friction depends upon the value of plasticity index, that is, it is affected by a number of factors. Furthermore, both dimensions of initial shear modulus and of shear strength are same. It is not without reason, therefore, to consider that initial shear modulus is closely related to shear strength.
    \end{enumerate}

    \switchcolumn

    \begin{enumerate}
        \item 根据实验室或现场试验的经验,在许多情况下,土壤刚度通常随强度而增加。
        \item \citet{Wilson2010419}通过实验室试验提供了许多有关杨氏模量$E$和不排水剪切强度$S_u$的数据。 此外,\citet{DAppolonia19711359}还从大量初始沉降曲线和不排水抗剪强度$S_u$获得的杨氏模量$E$的数据。 可以认识到,根据它们的数据计算出的比率$E/S_u$的值在一定范围内。
        \item 土的抗剪强度可以根据有效上覆压力,内摩擦角和内聚力的值来计算,内摩擦角取决于可塑性指数的值,即受多种因素影响。 此外,初始剪切模量和剪切强度的尺寸都相同。 因此,并非没有理由考虑初始剪切模量与剪切强度密切相关。
    \end{enumerate}

    \switchcolumn*
    
    The authors have tried, from the above mentioned reasons, to review the relation between initial shear modulus and shear strength of cohesive soils on the basis of the results of their own in-situ test (well-shooting test, standard penetration test) and laboratory test (triaxial compression test).

    \switchcolumn

    基于上述原因,作者试图根据他们自己的现场试验(射井试验,标准渗透试验)和实验室试验(三轴压缩试验)的结果来回顾粘性土的初始剪切模量与剪切强度之间的关系。


\end{paracol}